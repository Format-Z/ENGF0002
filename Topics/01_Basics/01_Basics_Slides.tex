\documentclass{beamer} % [aspectratio=169]
\usetheme{ucl}
\setbeamercolor{banner}{bg=darkred}
\setbeamersize{description width=2em}
\setbeamertemplate{navigation symbols}{\vspace{-2ex}} 

\usepackage[T1]{fontenc} % Turn £ into $
\usepackage{minted}
\usemintedstyle{emacs}


\usepackage{fancyvrb}
\usepackage{xcolor}
\usepackage{url}

\usepackage{natbib}
\usepackage{bibentry}
\usepackage{url}

\newenvironment{variableblock}[3]{%
  \setbeamercolor{block body}{#2}
  \setbeamercolor{block title}{#3}
  \begin{block}{#1}}{\end{block}}

\newcommand\emc[1]{\textcolor{midred}{\textbf{#1}}}

\AtBeginSection[]{
  \begin{frame}
  \vfill
  \centering
  \begin{beamercolorbox}[sep=8pt,center,shadow=true,rounded=true]{title}
    \usebeamerfont{title}\insertsectionhead\par%
  \end{beamercolorbox}
  \vfill
  \end{frame}
}

\author{Prof.\ Mark Handley \\ University College London, UK\\
\small{Based on slides from George Danezis}}
\title{Basics of Computer Programming \\ \& Software Engineering.}
\subtitle{ENGF0002: Design and Professional Skills }
% \institute{}
\date{Term 1, 2018}

\begin{document}
\nobibliography*

\frame{
\titlepage
}

\section{Introduction}

\frame{
\frametitle{Introducing Basics.}

In the `Basics' topic we will:
\begin{itemize}
\item Introduce basic Python \emc{language} concepts.
\item Illustrate \emc{algorithms}, to perform numerical computations.
\item Expose you to the problems of \emc{correctness}, and \emc{testing}.
\item Introduce \emc{good practices} when programming.
\end{itemize}

} 


\frame{
\frametitle{What is Computer Science \& Software Engineering?}

Computer Science:
\begin{itemize}
\item Studies the nature of \emc{information, computation, complexity} of algorithms, and their \emc{correctness}.
\item Deep mathematical foundations, including logic, algebra and probability theory.
\end{itemize}

Software Engineering:
\begin{itemize}
\item Studies how teams can repeatably \emc{build high quality, correct, usable and efficient software}, to meet \emc{people's needs}.
\item Programming is a foundational and important part of it.
\end{itemize}

The two are interlinked, and we will study them together.
} 

\frame{
\frametitle{Programming \& languages (I).}

Programming \& Programmability: 
\begin{itemize}
	\item A device is programmable if it allows a programmer (developer, engineer) to \emc{alter its behavior}. This is the act of programming.
	\item We usually think of a device as programmable, when the \emc{program itself takes the form or information}, rather than physical modification.
	\item A computer is the \emc{ultimate programmable device}, and can execute all computations. Simpler ones include video recording devices, microwave ovens, and alarm clocks.
	\item \emc{Programming is specializing} a device to solve \emc{a problem that people have}.
\end{itemize}
}

\frame{
\frametitle{Programming \& languages (II).}

Programming Languages:
\begin{itemize}
	\item The \emc{information describing the program} (code), is expressed in a programming language.
	\item \emc{Trivial} programming languages: button pressed to program an alarm clock. They are very \emc{low-level ways}, and lead to errors and inflexibility.
	\item Full computers are programmed in \emc{higher-level formal languages}. Programmers may express their intent directly, build complex abstractions, and compose programs from smaller fragments.
	\item Programming languages are formal languages but also \emc{human languages}. Programmers, use them to \emc{express and communicate their intent}.
\end{itemize}

}


\frame{
\frametitle{Why Python?} 
\begin{itemize}
	\item \emc{Real-world}, widely used programming language \\(4th in 2018 TIOBE ranking.)
	\item \emc{Multi-paradigm}: scripting, procedural, functional elements, object oriented, generics.
	\item Extensive \emc{eco-system} of tools and libraries. Great documentation.
	\item Significant \emc{industrial uses}.
	\item \emc{High-productivity}, and perfect for rapid prototyping.
\end{itemize}

\vspace{3mm}
A \emc{weakness} we will turn to a strenth:
\begin{itemize}
	\item It does not force you to adopt \emc{good programming practices}.
	\item It \emc{supports them}, but you must learn them!
\end{itemize}


}

\frame{
	
\frametitle{Other languages you might want to learn.}

\emc{Different problems} may require \emc{different tools}:
\begin{itemize}
\item Lower-level: Rust$*$, C++, \emc{C} --- COMP0002.
\item Static typing: Go$*$, \emc{Java}, C\# --- COMP0004.
\item Web: Javascript$*$.
\item Functional: \emc{Haskell}, Scala$*$ --- COMP0002.
\end{itemize}


\vspace{1mm}
$*$ - trendy languages at the moment

\vspace{3mm}
All programming languages mix and match a few principles. \\
Good computer scientists \emc{know those principles},\\ and can \emc{work in any language}.
}

\begin{frame}[fragile]
\frametitle{Using Python interactively.} 

\begin{itemize}
	\item Python 3.7. (\url{https://www.python.org/about/gettingstarted/})
	\item Open a command line console, and run the Python interpreter.
	\item Type your fist command: \mint{Python}{print("Hello World!")}
	\item You should see it executing!

\begin{Verbatim}[fontsize=\footnotesize]
mjh$ python3
Python 3.7.0 (default, Jun 28 2018, 05:55:06) 
[Clang 9.1.0 (clang-902.0.39.2)] on darwin
Type "help", "copyright", "credits" or "license" for more information.
>>> print("Hello World!")
Hello World!
>>> 
\end{Verbatim}

\end{itemize}

\end{frame}


\frame{
\frametitle{Key resources and tools.} 

The interactive interpreter is only good for quick experimentation.
\begin{itemize}
	\item \emc{Code editor}: atom (hip!), Sublime (hip!), Visual Studio Code. \\ Must haves: Good syntax highlighting, good handling of files \& folders, whitespace, looks very cool.
	\item \emc{Browser} with Python \emc{Documentation}: \url{https://docs.python.org/3/}. \\ (including Python \emc{tutorial} and \emc{Library reference}.)
	\item \emc{Command line} \& learn how to use it. \url{https://www.lynda.com/Linux-tutorials/Learn-Linux-Command-Line-Basics/435539-2.html}
	\item \emc{Stack Overflow} for Q\&A. \\ \url{https://stackoverflow.com/}
	\item \emc{Install pytest}. \url{https://docs.pytest.org/}
\end{itemize}
}

\frame{
\frametitle{Your physical \& mental well being.} 

\begin{itemize}
\item Approach the physical activity of programming with \emc{professionalism}.
\item Think of the \emc{ergonomics} of your physical environment: chair height, desk, posture, keyboard style, monitor type and positioning.
\item Make sure you \emc{enjoy your working environment}: light, sounds, distractions.
\item \emc{Take breaks}, at least every hour.
\end{itemize}

}

\begin{frame}[fragile]
\frametitle{The `Hello World!' program in Python.} 

Programs live in files. A simple Python program:

	\inputminted[
		xleftmargin=1.4em,
		frame=lines,
		framesep=2mm,
		%baselinestretch=1.2,
		bgcolor=stone,
		fontsize=\footnotesize,
		linenos
	]{python}{src/hello_world.py}

Execute your first program, by running:
\begin{Verbatim}
$ python3 hello_world.py
Hello World!
\end{Verbatim}

\end{frame}

\begin{frame}
\frametitle{Comments}

Comments are not executed by Python, but form part of your program.
\begin{itemize}
\item Comments \emc{communicate intent} to your future self and others.
\item Express in comments concepts you cannot directly express in code.
\item Prefer to express concepts directly in code.
\item Keep comments local to the relevant code.
\item Do not overcomment; avoid duplicate, redundant or wrong comments. 
\item Every comment will have to be maintained in the future.
\end{itemize}
\end{frame}

\begin{frame}
\frametitle{Whitespace.}

Python is \emc{sensitive to indentation} using whitespace. Use 3 or 4 spaces (not a tab) to indent blocks of code. More about this later \ldots.

	\inputminted[
		xleftmargin=1.4em,
		frame=lines,
		framesep=2mm,
		%baselinestretch=1.2,
		bgcolor=stone,
		fontsize=\footnotesize,
		linenos
	]{python}{src/hello_world.py}

\end{frame}

\begin{frame}
\frametitle{How do we know a program does what it should?}

\emc{Correctness is the most important problem} in software engineering and computer science!

\vspace{3mm}
\emc{Testing is the key technique} to produce high quality, correct programs. It is an activity that is continuous, and performed in parallel with programming.

\vspace{3mm}
Complimentary techniques for ensuring program correctness include \emc{formal verification}. Those are more expensive, but necessary to reason about very complex problems.

\vspace{3mm}
Finally, formal periodic \emc{code reviews}, or continuous reflection through \emc{pair programming} also improve quality.

\end{frame}


\frame{
\frametitle{Test Code Continuously.} 

\emc{Testing} is the most important technique to gain confidence a program does what it should:
\begin{itemize}
	\item Unit Test every 2-5 lines of code you write. Think how to \emc{test every snippet of code} just before, while, or just after you write it.
	\item Use a mature \emc{tool for testing} your programs. We will use \texttt{pytest}. \url{https://docs.pytest.org/}.
	\item \emc{Different testing techniques} for code correctness, integration, security, performance and user experience.
	\item Shortcomings of testing: \emc{lack of completeness}.
\end{itemize}

}

\frame{
\frametitle{Testing the `Hello World!' program.} 

\emc{Tests are snippets of code} executing parts of your program. \\ Eg.\ the test of the simple Python program is:

	\inputminted[
		xleftmargin=1.4em,
		frame=lines,
		framesep=2mm,
		%baselinestretch=1.2,
		bgcolor=stone,
		fontsize=\footnotesize,
		linenos
	]{python}{src/test_hello_world.py}

We run the \emc{test suite}, by executing \texttt{pytest test\_hello\_world.py} on the command line.


}

\frame{
\frametitle{Code documentation and its tests. } 

\emc{Code documentation} is not executed, but forms part of the program. It is useful to your future self, or others that want to use part of your program. Hence, \emc{document reusable units of code}.

	\inputminted[
		xleftmargin=1.4em,
		highlightlines={6-7},
		frame=lines,
		framesep=2mm,
		%baselinestretch=1.2,
		bgcolor=stone,
		fontsize=\footnotesize, 
		linenos
	]{python}{src/hello_world_doc.py}

You must \emc{test code in documentation}. \\ \texttt{pytest -vs --doctest-modules test\_hello\_world.py}.

}


\begin{frame}[fragile]

\frametitle{Example of a successful test run. }

Executing \texttt{pytest} runs all tests in functions named \texttt{test\_*} and documentation strings.

\begin{Verbatim}[fontsize=\scriptsize]
$ pytest -vs --doctest-modules src/*hello*.py
=================================== test session starts ====================================
platform darwin -- Python 3.7.0, pytest-3.7.4, py-1.6.0, pluggy-0.7.1 -- /opt/local/Library/Frameworks/Python.framework/Versions/3.7/bin/python3.7
cachedir: .pytest_cache
rootdir: /Users/mjh/teaching/engf0002/Design_and_Professional_Skills/Topics/01_Basics, inifile:
collected 2 items                                                                          

src/hello_world_doc.py::hello_world_doc.hello_world PASSED
src/test_hello_world.py::test_hello_world PASSED

================================= 2 passed in 0.02 seconds =================================
\end{Verbatim}

\end{frame}

\frame{
\frametitle{Summary and next steps.}

The rhythm-of-the-programming-business:
\begin{itemize}
	\item \emc{Think} of the problem $\rightarrow$ \emc{Code} feature (2-5 lines) $\rightarrow$ \emc{Write test} for feature $\rightarrow$ Run \emc{all} tests $\rightarrow$ Fix until \emc{all tests pass} $\rightarrow$ Think \ldots
	\item Working programmers \emc{run tests a few times per minute}.
	\item Testing practice imposes an \emc{incremental approach} to software building.
	\item Professional standards: aspire to deliver programs with \emc{zero bugs}. 
	Remember that \emc{`bugs' may cost millions or even kill}.
\end{itemize}

\vspace{5mm}
Still to cover: functions and function calls (\texttt{def}, \texttt{()}), modules (\texttt{from}, \texttt{import}) and a lot more python \ldots

}

\section{Algorithms}

\begin{frame}
\frametitle{What is an `algorithm'?} 

An algorithm describes a sequence of steps leading to the solution of a computational problem.

\vspace{5mm}
You are familiar with a number of algorithms from maths:
\begin{itemize}
\item How to perform long multiplication and division with a pencil and paper.
\item How to find the roots of a quadratic equation $ax^2+bx+c = 0$.
\item How to expand brackets for $(a+b)(c+d) = ?$
\end{itemize}

\begin{block}{Beyond maths \ldots}
Once we express wider parts of the world as information, algorithms can solve more exciting tasks: eg.\ rendering a 3D scene in a computer game, encrypt communications, build interactive social networks on-line, and program cars to drive autonomously.
\end{block}

\end{frame}

\begin{frame}
  \frametitle{Lifts}

  \begin{block}{Problem Statement}
    A building has a lift.\\
    Inside the lift are buttons to choose floors.\\
    Outside the lift are buttons to call the lift.\\
    {\bf How should the lift choose which floor to go to next?}
  \end{block}
  \centering
  \includegraphics[height=40mm]{img/elevator.jpg}
\end{frame}

\begin{frame}
  \frametitle{Lifts}
  \begin{block}{Algorithm 1}
    Go to the nearest floor that is selected.
  \end{block}
  \centering
  \includegraphics[height=40mm]{img/elevator.jpg}
\end{frame}

\begin{frame}
  \vspace{-16mm}
  \begin{block}{Algorithm 1}
    Go to the nearest floor that is selected.
  \end{block}
  \vspace{5mm}
  \begin{variableblock}{Bug report}{bg=stone,fg=black}{bg=black,fg=white}
    \begin{itemize}
    \item Under certain workloads, people in the lift starve to death before they reach the destination floor.
    \end{itemize}
  \end{variableblock}
\end{frame}

\begin{frame}
  \vspace{-16mm}
  \begin{block}{Algorithm 1}
    Go to the nearest floor that is selected.
  \end{block}
  \centering
  \includegraphics[width=120mm, trim=0mm 0mm 0mm 50mm, clip=true]{img/lift2-anim1.pdf}

\end{frame}

\begin{frame}
  \vspace{-16mm}
  \begin{block}{Algorithm 1}
    Go to the nearest floor that is selected.
  \end{block}
  \centering
  \includegraphics[width=120mm, trim=0mm 0mm 0mm 50mm, clip=true]{img/lift2-anim2.pdf}

\end{frame}

\begin{frame}
  \vspace{-16mm}
  \begin{block}{Algorithm 1}
    Go to the nearest floor that is selected.
  \end{block}
  \centering
  \includegraphics[width=120mm, trim=0mm 0mm 0mm 50mm, clip=true]{img/lift2-anim3.pdf}

\end{frame}

\begin{frame}
  %\vspace{-16mm}
  \begin{block}{Algorithm 1}
    Go to the nearest floor that is selected.
  \end{block}
  \vspace{5mm}
  \begin{variableblock}{Bug report}{bg=stone,fg=black}{bg=black,fg=white}
    \begin{itemize}
    \item Under certain workloads, people in the lift starve to death before they reach the destination floor.
    \end{itemize}
  \end{variableblock}
  \vspace{5mm}
  \begin{block}{Diagnosis}
    Need to prioritse going to the floor selected by people in the lift.
  \end{block}
\end{frame}

\begin{frame}
  \frametitle{Lifts}
  \begin{block}{Algorithm 2}
    \begin{itemize}
    \item Go to the nearest floor selected by people in the lift.
    \item If no floor selected in the lift, go to nearest floor where someone has pressed a call button.
    \end{itemize}
  \end{block}
  \centering
  \includegraphics[height=40mm]{img/elevator.jpg}
\end{frame}

\begin{frame}
  \begin{block}{Algorithm 2}
    \begin{itemize}
    \item Go to the nearest floor selected by people in the lift.
    \item If no floor selected in the lift, go to nearest floor where someone has pressed a call button.
    \end{itemize}
  \end{block}
  \vspace{5mm}
  \begin{variableblock}{Bug report}{bg=stone,fg=black}{bg=black,fg=white}
    \begin{itemize}
    \item Under certain workloads, people waiting for the lift starve to death because the lift always goes past their floor.
    \end{itemize}
  \end{variableblock}
\end{frame}


\begin{frame}
  \vspace{-10mm}
  \begin{block}{Algorithm 2}
    \begin{itemize}
    \item Go to the nearest floor selected by people in the lift.
    \item If no floor selected in the lift, go to nearest floor where someone has pressed a call button.
    \end{itemize}
  \end{block}
  \centering
  \includegraphics[width=120mm, trim=0mm 0mm 0mm 50mm, clip=true]{img/lift2-anim4.pdf}

\end{frame}

\begin{frame}
  \vspace{-10mm}
  \begin{block}{Algorithm 2}
    \begin{itemize}
    \item Go to the nearest floor selected by people in the lift.
    \item If no floor selected in the lift, go to nearest floor where someone has pressed a call button.
    \end{itemize}
  \end{block}
  \centering
  \includegraphics[width=120mm, trim=0mm 0mm 0mm 50mm, clip=true]{img/lift2-anim5.pdf}
\end{frame}

\begin{frame}
  \vspace{-10mm}
  \begin{block}{Algorithm 2}
    \begin{itemize}
    \item Go to the nearest floor selected by people in the lift.
    \item If no floor selected in the lift, go to nearest floor where someone has pressed a call button.
    \end{itemize}
  \end{block}
  \centering
  \includegraphics[width=120mm, trim=0mm 0mm 0mm 50mm, clip=true]{img/lift2-anim6.pdf}
\end{frame}

\begin{frame}
  \vspace{-2mm}
  \begin{block}{Algorithm 2}
    \begin{itemize}
    \item Go to the nearest floor selected by people in the lift.
    \item If no floor selected in the lift, go to nearest floor where someone has pressed a call button.
    \end{itemize}
  \end{block}
  \vspace{1mm}
  \begin{variableblock}{Bug report}{bg=stone,fg=black}{bg=black,fg=white}
    \begin{itemize}
    \item Under certain workloads, people in the lift starve to death before they reach the destination floor.
    \end{itemize}
  \end{variableblock}
  \vspace{1mm}
  \begin{block}{Diagnosis}
    Should not skip a floor where someone wants to go in the direction the lift is going.
  \end{block}
\end{frame}

\begin{frame}
  \begin{block}{Algorithm 3}
    \begin{itemize}
      \item Go to the next floor selected by people in the lift, but don't go past a floor where someone wants to go in the same direction the lift is going.
      \item If no floor is selected, go to the nearest floor where someone has pressed a button.
    \end{itemize}
  \end{block}
  \centering
  \includegraphics[height=40mm]{img/elevator.jpg}
\end{frame}

\begin{frame}
  \begin{block}{Algorithm 3}
    \begin{itemize}
      \item Go to the next floor selected by people in the lift, but don't go past a floor where someone wants to go in the same direction the lift is going.
      \item If no floor is selected, go to the nearest floor where someone has pressed a button.
    \end{itemize}
  \end{block}
  \vspace{5mm}
  \begin{variableblock}{Bug report}{bg=stone,fg=black}{bg=black,fg=white}
    \begin{itemize}
    \item Under certain workloads, people in the lift starve to death before they reach the destination floor.
    \end{itemize}
  \end{variableblock}
\end{frame}

\begin{frame}
  \vspace{-11mm}
  \begin{block}{Algorithm 3}
    \begin{itemize}
      \vspace{-1mm}
      \item \small Go to the next floor selected by people in the lift, but don't go past a floor where someone wants to go in the same direction the lift is going.
      \vspace{-1mm}
      \item \small If no floor is selected, go to the nearest floor where someone has pressed a button.
      \vspace{-1mm}
    \end{itemize}
  \end{block}
  \centering
  \includegraphics[width=120mm, trim=0mm 0mm 0mm 50mm, clip=true]{img/lift2-anim7.pdf}

\end{frame}

\begin{frame}
  \vspace{-11mm}
  \begin{block}{Algorithm 3}
    \begin{itemize}
      \vspace{-1mm}
      \item \small Go to the next floor selected by people in the lift, but don't go past a floor where someone wants to go in the same direction the lift is going.
      \vspace{-1mm}
      \item \small If no floor is selected, go to the nearest floor where someone has pressed a button.
      \vspace{-1mm}
    \end{itemize}
  \end{block}
  \centering
  \includegraphics[width=120mm, trim=0mm 0mm 0mm 50mm, clip=true]{img/lift2-anim8.pdf}

\end{frame}

\begin{frame}
  \vspace{-11mm}
  \begin{block}{Algorithm 3}
    \begin{itemize}
      \vspace{-1mm}
      \item \small Go to the next floor selected by people in the lift, but don't go past a floor where someone wants to go in the same direction the lift is going.
      \vspace{-1mm}
      \item \small If no floor is selected, go to the nearest floor where someone has pressed a button.
      \vspace{-1mm}
    \end{itemize}
  \end{block}
  \centering
  \includegraphics[width=120mm, trim=0mm 0mm 0mm 50mm, clip=true]{img/lift2-anim9.pdf}

\end{frame}

\begin{frame}
  \begin{block}{Algorithm 3}
    \begin{itemize}
      \item Go to the next floor selected by people in the lift, but don't go past a floor where someone wants to go in the same direction the lift is going.
      \item If no floor is selected, go to the nearest floor where someone has pressed a button.
    \end{itemize}
  \end{block}
  \vspace{5mm}
  \begin{variableblock}{Bug report}{bg=stone,fg=black}{bg=black,fg=white}
    \begin{itemize}
    \item Under certain workloads, people in the lift starve to death before they reach the destination floor.
    \end{itemize}
  \end{variableblock}
  \vspace{1mm}
  \begin{block}{Diagnosis}
    Lift should not turn round if it's heading towards people who are waiting.
  \end{block}
\end{frame}

\begin{frame}
  \begin{block}{Algorithm 4}
    \begin{itemize}
    \item Go in one direction, stopping at floors where:
      \begin{itemize}
      \item people outside want to go in that direction, or
      \item where people in the lift want to go in that direction.
        \end{itemize}
       \item Change direction only when there are no more selected floors or call buttons pressed in that direction.
    \end{itemize}
  \end{block}
  \centering
  \includegraphics[height=40mm]{img/elevator.jpg}
\end{frame}

  
\begin{frame}
  \begin{block}{Algorithm 4}
    \begin{itemize}
    \item Go in one direction, stopping at floors where:
      \begin{itemize}
      \item people outside want to go in that direction, or
      \item where people in the lift want to go in that direction.
        \end{itemize}
       \item Change direction only when there are no more selected floors or call buttons pressed in that direction.
    \end{itemize}
  \end{block}
  \vspace{5mm}
  \begin{variableblock}{Bug reports}{bg=stone,fg=black}{bg=black,fg=white}
    \begin{itemize}
    \item --- NO BUG REPORTS FILED ---
    \end{itemize}
  \end{variableblock}
\end{frame}

  
\begin{frame}
  \begin{block}{Challenge 1}
    \begin{itemize}
    \item What should the lift do if no call button is pressed?
    \end{itemize}
  \end{block}
  \centering
  \begin{block}{Challenge 2}
    \begin{itemize}
    \item Extend the algorithm so it works efficiently to control two lifts.
    \end{itemize}
  \end{block}
  \centering
  \includegraphics[height=40mm]{img/elevator.jpg}
\end{frame}

\bibliographystyle{alpha}
\nobibliography{references}

\end{document}
