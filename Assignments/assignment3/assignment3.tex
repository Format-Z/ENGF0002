\documentclass{article}

\usepackage{graphicx}
\usepackage{amsmath}	
	
\title{Design \& Professional Skills\\
  Assignment 3: Debugging the {\em Frogger} Game}
\author{}
\date{}
	
\begin{document}

\maketitle

\section*{Instructions}

In the {\tt mhandley/ENGF0002} github repository, in the {\tt
  Assignments/assignment3} directory, there are a number of python and
PNG files that together comprise an implementation of the 1980s video
game \textit{Frogger}.  The game is written in an object oriented
style, and roughly conforms to a Model/View/Controller design pattern,
as discussed in class.  Run the game by typing:\\ \texttt{python3
  frogger.py}\\ or the equivalent on your system.  On the lab
Linux machines, a new version of python 3 is in
\texttt{/opt/Python/Python-3.7/bin/python}.  The game does work using
the departmental Linux machines using CSRW, though it does jitter a
bit.


The objective of the game is to cross the road, avoiding the cars,
cross the river by jumping on the logs or turtles, and make it to one
of the five frog homes at the top of the screen.  You start out with
seven lives and a certain amount of time.  You must get frogs to all
five homes before the time runs out, or it's Game Over.  If you
succeed in filling all five homes, you get to move on to the next
level, and the game gets a little harder.

The game requires python 3 and a recent enough version of tkinter that has PNG support.

The problem is that the game has at least five bugs that make it unplayable.  Your task:
\begin{enumerate}
\item Play the game.
\item Find a bug.
\item Write a brief bug report describing the bug.
\item Identify the cause of the bug.  Write a brief summary of the cause.
\item Fix the bug.
\item Repeat from 1.
\end{enumerate}

\subsection*{Bug Reports}

A bug report should be brief and to the point.  It should include:
\begin{itemize}
\item One sentence summary of the bug.
\item Description of what happens.
\item Description of what you think should happen.
\item Instructions for how to reproduce the bug.
\end{itemize}

\subsection*{Understanding the bug}

Once you've written the bug report, look at the code.

Identify what the code is doing when the bug is triggered.  Sometimes
the cause may be obvious from reading the code.  Often the cause is
not obvious, and even the flow of the code may not be obvious.  Then
you will need to instrument the code to figure out what it is doing.
In this case, I just want you to instrument the code using
\texttt{print()} - there's no need to use a debugger.  Generally, you
want to instrument the code without changing its behaviour until you
gather enough information to understand what the code is actually
doing (and hence why it differs from what it should be doing).  

You may however want to temporarily change the code to make it easier
to reproduce a bug, and then revert those changes after you've fixed
the bug. A common debugging technique is to reduce the code complexity
by removing code to reduce it to the simplest case that still exhibits
the buggy behaviour.  This is a valuable technique when a bug is hard
to reproduce.

In general, you're hunting for evidence until you've found out what
the program is actually doing.  Only when you understand what the
program is doing should you think about how to fix it.

\subsection*{Fixing the bugs}

These bugs are very simple.  Some are one line fixes, none requires
more than about three lines of extra code to fix.  One will require you read the code very carefully.

\subsection*{Assessment}

This assignment is not assessed.  The purpose of the assignment is to
give you practice reading, understanding, and debugging a non-trivial
piece of code.  You many work with your friends on this assignment if
you wish.  You must hand in a reasonable attempt via Moodle to receive
a binary mark, as we want to see how you are progressing.  Hand in a
zipfile containing the text of your bug reports, your brief
explanations of the causes of the bugs, and the python source code of
your fixed version of the game.

\subsection*{Optional Extra}

A few members of the class are more experienced programmers.  If you
find fixing the bugs easy, once you've fixed them, consider extending
the game.  For example, in the original 1980s game, the turtles submerge
every so often, and there are crocodiles and other hazards.  Also the
way the game gets harder with each level is somewhat better.

There will be a separate submission area on Moodle for game
extensions.  You won't get any extra marks, but if we like your code
enough to use next year, we'll figure out some reward.

\end{document}
